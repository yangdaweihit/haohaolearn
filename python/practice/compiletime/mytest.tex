\documentclass{article}
\usepackage{xeCJK}
% \usepackage[UTF8]{ctex}
\begin{document}
\LaTeX{}
你好
\section{主梁截面几何特性计算}

后张法预应力混凝土截面几何特性应根据不同的受力阶段分别计算。本示例中的T
形梁从施工到运营经历了如下3个阶段。

  浆, 所以其截面特性为计入非预应力钢筋影响(将非预应力钢筋换算为混凝土)
  的净截面。该截面的截面特性计算中应扣除预应力管道的影响, T梁翼板宽度
后张法预应力混凝土截面几何特性应根据不同的受力阶段分别计算。本示例中的T
形梁从施工到运营经历了如下3个阶段。

  浆, 所以其截面特性为计入非预应力钢筋影响(将非预应力钢筋换算为混凝土)
  的净截面。该截面的截面特性计算中应扣除预应力管道的影响, T梁翼板宽度

后张法预应力混凝土截面几何特性应根据不同的受力阶段分别计算。本示例中的T
形梁从施工到运营经历了如下3个阶段。

  浆, 所以其截面特性为计入非预应力钢筋影响(将非预应力钢筋换算为混凝土)
  的净截面。该截面的截面特性计算中应扣除预应力管道的影响, T梁翼板宽度
后张法预应力混凝土截面几何特性应根据不同的受力阶段分别计算。本示例中的T
形梁从施工到运营经历了如下3个阶段。

  浆, 所以其截面特性为计入非预应力钢筋影响(将非预应力钢筋换算为混凝土)
  的净截面。该截面的截面特性计算中应扣除预应力管道的影响, T梁翼板宽度
后张法预应力混凝土截面几何特性应根据不同的受力阶段分别计算。本示例中的T
形梁从施工到运营经历了如下3个阶段。

  浆, 所以其截面特性为计入非预应力钢筋影响(将非预应力钢筋换算为混凝土)
  的净截面。该截面的截面特性计算中应扣除预应力管道的影响, T梁翼板宽度
后张法预应力混凝土截面几何特性应根据不同的受力阶段分别计算。本示例中的T
形梁从施工到运营经历了如下3个阶段。

  浆, 所以其截面特性为计入非预应力钢筋影响(将非预应力钢筋换算为混凝土)
  的净截面。该截面的截面特性计算中应扣除预应力管道的影响, T梁翼板宽度

后张法预应力混凝土截面几何特性应根据不同的受力阶段分别计算。本示例中的T
形梁从施工到运营经历了如下3个阶段。

  浆, 所以其截面特性为计入非预应力钢筋影响(将非预应力钢筋换算为混凝土)
  的净截面。该截面的截面特性计算中应扣除预应力管道的影响, T梁翼板宽度
后张法预应力混凝土截面几何特性应根据不同的受力阶段分别计算。本示例中的T
形梁从施工到运营经历了如下3个阶段。

  浆, 所以其截面特性为计入非预应力钢筋影响(将非预应力钢筋换算为混凝土)
  的净截面。该截面的截面特性计算中应扣除预应力管道的影响, T梁翼板宽度

\end{document}
%%% Local Variables:
%%% mode: latex
%%% TeX-master: t
%%% End:
